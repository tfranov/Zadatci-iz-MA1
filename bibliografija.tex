\newpage
\fancyhead[RO, RE]{}
\fancyhead[LO, LE]{}
\renewcommand{\headrulewidth}{0pt}
\begin{thebibliography}{20}
\bibitem{1}
M. Vuković, \textit{Matematička logika}, skripta, PMF -- Matematički odsjek, Zagreb, 2007.
\bibitem{2}
K. Horvatić, \textit{Linearna algebra}, Golden marketing -- Tehnička knjiga, Zagreb, 2004.
\bibitem{3}
D. Krizmanić, \textit{Matematička analiza 1}, skripta, Fakultet za matematiku Sveučilišta u Rijeci, Rijeka, 2024.
\bibitem{4}
A. Dujella, \textit{Teorija brojeva}, Školska knjiga, Zagreb, 2019.
\bibitem{5}
S. Kurepa, \textit{Matematička analiza 1}, Školska knjiga, Zagreb, 1976.
\bibitem{6}
\textit{Applying the Arithmetic Mean Geometric Mean Inequality}. Brilliant.org. Preuzeto u 11:12, 19. srpnja 2024., \href{https://brilliant.org/wiki/applying-the-arithmetic-mean-geometric-mean/}{Za link kliknite ovdje}.
\bibitem{7}
\textit{Arithmetic Mean - Geometric Mean}. Brilliant.org. Preuzeto u 15:54, 19. srpnja 2024., \href{https://brilliant.org/wiki/arithmetic-mean-geometric-mean/}{Za link kliknite ovdje}.
\bibitem{8}
B. P. Demidovič i suradnici, \textit{Zadaci i riješeni primjeri iz matematičke analize za tehničke fakultete}, Sedmo ispravljeno izdanje, Golden marketing, Zagreb, 2003.
\bibitem{9} S. Mardešić, \textit{Matematička analiza u $n$-dimenzionalnom realnom prostoru}, Prvi dio, Školska knjiga, Zagreb, 1974.
\bibitem{10} Grupa autora, \foreignlanguage{russian}{Spravočnoe posobie po matematičeskomu analizu, Prvi dio, Višča škola, Kijev, 1978.}
\bibitem{11} M. Vuković, \textit{Teorija skupova}, predavanja, PMF -- Matematički odsjek, Zagreb, 2015.
\bibitem{12} B. Pavković, D. Veljan, \textit{Elementarna matematika 1}, Školska knjiga, Zagreb, 2004.
\bibitem{13} T.-L. Radulescu, V. D. Radulescu, T. Andreescu, \textit{Problems in Real Analysis: Advanced Calculus on the Real Axis}, Springer, 2009.
\bibitem{14} S. Kurepa, \textit{Matematička analiza 2}, Školska knjiga, Zagreb, 1997.
\bibitem[15] GGrupa autora, \textit{Male teme iz matematike}, Element d.o.o., Zagreb, 1994.
\end{thebibliography}
Dio zadataka preuzet je iz kolokvija iz \textit{Matematičke analize 1} na PMF-u.