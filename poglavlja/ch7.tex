\section{Definicija derivacije}
\begin{definition}[Definicija neprekidnosti funkcije]
Neka je $I\subseteq \mathbb{R}$ otvoreni interval i $c\in I$. Kažemo da je funkcija $f : I\to \mathbb{R}$ \textbf{neprekidna u točki $c$} ako za svaki $\epsilon>0$ postoji $\delta>0$ takav da za sve $x\in I$ vrijedi 
$$|x-c|<\delta\Rightarrow |f(x)-f(c)|<\epsilon.$$
\begin{itemize}
\item $f$ je \textbf{neprekidna na skupu} $S\subseteq I$ ako je ona neprekidna u svakoj točki skupa $S$.
\item $f$ ima \textbf{prekid} u točki $c$ ako ona nije neprekidna u $c$.
\item $f$ ima \textbf{prekid na skupu} $S\subseteq I$ ako postoji bar jedna točka $c\in S$ u kojoj $f$ ima prekid.
\end{itemize}
\end{definition}

